\subsection{Módulo de actualizaciones}

El desarrollo del proyecto se organizó en módulos, cada uno con objetivos específicos y alcanzables para asegurar avances concretos en cada etapa.\par

En la fase inicial, nos enfocamos en una tarea esencial en cualquier proyecto de software: mantener la compatibilidad con las diferentes versiones del sistema operativo. De esta forma, obtuvimos beneficios significativos en términos de estabilidad y compatibilidad, resultando en un sistema más confiable. Además, estando alineados con la ultima versión del mismo, nos acerca a la posibilidad de contribuir al proyecto de una manera directa.\par

El trabajo comenzó con la versión 11 de FreeBSD, basado en el trabajo integrador previo realizado por Nicolás Papp y Tomás Turina, y se extendió hasta la última versión estable, la 13.1.\par

Para gestionar las actualizaciones de manera ordenada y efectiva, así como también, comprender mejor los cambios entre versiones; el módulo se dividió en tres etapas progresivas: actualización a la última \textit{release} de la versión 11, y actualizaciones a las versiones 12 y 13 de FreeBSD.\par


\subsubsection{Actualización al último release de la \textit{v11}}

En esta primera iteración se buscó llevar los cambios del proyecto integrador previo realizados en la versión 11.0.0 del sistema operativo FreeBSD a la última release de ésta versión, es decir, al 11.4.0.\par

Durante dicha actualización, nos encontramos con un cambio significativo que tuvo un impacto relevante en nuestra tesis: la modificación en la función \textit{maybe\_preempt}. En la versión anterior, esta función se encargaba de determinar si un nuevo hilo debía tomar el control del procesador inmediatamente, reemplazando al hilo actual en ejecución. En caso afirmativo, se realizaba el cambio de contexto y comenzaba su ejecución, sustituyendo al hilo que se encontraba en el procesador.\par

En la versión más reciente, se introdujo una nueva estrategia, posponer el cambio de contexto para un momento más adecuado mediante un sistema de banderas. En lugar de realizar el cambio de contexto de manera inmediata, se implementó un mecanismo donde el hilo recién encolado establece el valor de la bandera \textit{td\_owepreempt} en 1, indicando la necesidad de desalojar al que se encuentra en ejecución, para así tomar el control.\par

Si bien el metodo de preemption no se encuentra modelado en nuestra Red de Petri de Recursos, este cambio impactó en el código del planificador de nuestro trabajo ya que implicó la remoción de algunos disparos de transiciones que se provocaban en el cambio de contexto inmediato. Con la nueva estrategia, fue necesario ajustar el código de nuestra Red de Petri para reflejar el cambio en la política de planificación.\par

\subsubsection{Actualización a la versión 12}

Al comparar las diferencias entre las versiones 11.4.0 y 12.3.0 (último \textit{release}) de FreeBSD, no se encontraron cambios significativos en los archivos relevantes para nuestro proyecto del planificador. Debido a esto, no fue necesario realizar ajustes específicos para esta versión.


\subsubsection{Actualización a versión 13}

La versión 13 trajo con ella varios cambios significativos en el código del planificador. Los cambios principales, y que mayor tiempo de análisis e investigación tomaron, fueron los cambios dentro de la función \textit{sched\_switch()}, función encargada de realizar el cambio de contexto entre hilos en el planificador 4BSD.\par

El primero de ellos (ver apéndice \ref{appendix:apA1}) ya se encuentra en la firma de la función y sus variables: previamente, \textit{newtd} era un parámetro de \textit{sched\_switch()} que se utilizaba para indicar explícitamente el hilo que se iba a ejecutar a continuación. Sin embargo, en la versión 13, el parámetro \textit{newtd} se eliminó y con ello se simplificó la lógica de selección del hilo (ver apéndice \ref{appendix:apA3}), teniendo que ajustar algunos disparos en la Red de Petri que previamente eran necesarios para mantener la coherencia de los recursos relacionados, con estos desencolados en el planificador.\par

En la versión 13 de FreeBSD, se realizó una reorganización significativa de los locks en la función \textit{sched\_switch()} (ver apéndice \ref{appendix:apA2}). Estos cambios introdujeron comportamientos inesperados en la red, como el disparo de transiciones no sensibilizadas que desajustaban el estado de la red y causaban fallos en el sistema. Después de depurar el código, identificamos que el problema principal era el manejo incorrecto de la transición RETURN\_VOL. Específicamente, el método \textit{resource\_expulse\_thread()} estaba disparando esta transición sin bloquear previamente el \textit{td\_lock}. Sin este bloqueo, no era posible cambiar el estado del hilo, lo que impedía que regresara al estado de \textit{CAN\_RUN} o listo para ejecutarse. Como resultado, ningún hilo podía ser ejecutado por los procesadores más de una vez, lo que bloqueaba el sistema. Tras realizar los ajustes necesarios, logramos resolver este problema y garantizar el correcto funcionamiento de la red.\par

\subsubsection{Resultados}

Durante la actualización a las diferentes versiones del sistema operativo FreeBSD, se logró garantizar el funcionamiento correcto de todas las versiones, lo cual fue un logro significativo. Esto permitió no solo mantener el proyecto al día, sino también adentrarnos en el código del sistema operativo y comprender su funcionamiento de manera más integral.\par

Se tuvo la oportunidad de leer y analizar el código fuente en profundidad. Esto nos permitió adquirir un conocimiento detallado del proyecto y nos familiarizó con su estructura y características particulares. Esta inmersión en el código nos brindó una base sólida para abordar tareas posteriores de manera más eficiente y efectiva.\par

Adicionalmente, durante el proceso de actualizaciones, se identificaron, realizaron y documentaron algunas mejoras respecto a versiones anteriores del proyecto. Dichas mejoras fueron registradas en el repositorio de documentación (apartado \ref{sect:wiki}); y algunas de ellas, en el presente documento, con el objetivo de mantener un registro detallado y accesible para trabajos futuros.\par

\subsubsection{Próximos pasos}

Como se desarrolló previamente, sostenemos la importancia de mantener el proyecto actualizado con las últimas versiones del sistema operativo FreeBSD para asegurar su eficiencia, seguridad y compatibilidad con tecnologías mas recientes.

Para lograr este objetivo, es esencial monitorear continuamente las actualizaciones y aplicar los cambios necesarios de manera oportuna. Esto evitará el rezago respecto a \textit{releases} más recientes y garantizará que el proyecto se beneficie de las últimas mejoras y correcciones de errores.

Mantener el software actualizado en proyectos de código abierto como FreeBSD no solo mejora el rendimiento y la seguridad, sino que también facilita una integración más estrecha con la comunidad de desarrolladores y usuarios. Estar al día con las actualizaciones permite recibir retroalimentación valiosa, contribuyendo al desarrollo colaborativo.

Para futuros trabajos en este proyecto, se recomienda priorizar la importancia de este aspecto y comprometerse a dedicar los recursos y el tiempo necesarios para llevar a cabo las actualizaciones de manera regular.