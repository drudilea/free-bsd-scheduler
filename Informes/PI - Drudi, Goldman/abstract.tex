\begin{Large}
    \textbf{Resumen}
    \vspace{.5cm}
\end{Large}

El sistema operativo se encarga de gestionar los recursos de hardware de un dispositivo electrónico y de brindar servicios a los programas de aplicación. La gestión de los procesos e hilos y los recursos de un sistema operativo son fundamentales para su correcto funcionamiento y desempeño. En este contexto, la planificación es una tarea crítica para asegurar el uso eficiente de los recursos del sistema.\par

En este trabajo de tesis, se parte del modelado y la optimización del planificador a corto plazo mediante Redes de Petri realizado en un proyecto integrador previo por dos alumnos de la FCEFyN. Las Redes de Petri son una herramienta matemática formal de modelado que permite representar y analizar sistemas de eventos discretos, como es el caso de la planificación de procesos.\par

En primer instancia, planteamos adaptar el trabajo del proyecto integrador previo,  a diferentes versiones del sistema operativo. Esto es importante ya que las versiones anteriores eventualmente dejan de tener soporte por parte de la comunidad, así como también, las funcionalidades y la seguridad evolucionan con las nuevas actualizaciones.   Planteadas las actualizaciones correspondientes, se presentan propuestas sobre integración de nuevas funcionalidades de gestión de procesadores en el planificador. Los procesadores son un recurso valioso en un sistema operativo, y su correcta administración puede tener un impacto significativo en el desempeño del sistema. El uso de las Redes de Petri para modelar estas funcionalidades, nos permitirá comprobar con mayor facilidad la ausencia problemas de concurrencia muy comunes en la planificación, así como también aportarán a la hora de captar los estados y eventos del planificador.\par
