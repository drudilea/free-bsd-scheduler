\begin{Large}
    \textbf{Resumen}
    \vspace{.5cm}
\end{Large}

El sistema operativo se encarga de gestionar los recursos de hardware de un dispositivo electrónico y de brindar servicios a los programas de aplicación. La gestión de los procesos e hilos y los recursos de un sistema operativo son fundamentales para su correcto funcionamiento y desempeño. En este contexto, la planificación es una tarea crítica para asegurar el uso eficiente de los recursos del sistema.\par

En este trabajo de tesis, se parte del modelado y la optimización del planificador a corto plazo mediante Redes de Petri realizado en un proyecto integrador previo por dos alumnos de la FCEFyN. Las Redes de Petri son una herramienta matemática formal de modelado que permite representar y analizar sistemas de eventos discretos, como es el caso de la planificación de procesos. El uso de ellas, nos permitirá comprobar con mayor facilidad la ausencia de problemas de concurrencia muy comunes en la planificación, así como también aportarán a la hora de captar los estados y eventos del planificador.\par

En el marco de esta investigación, comenzamos con la adaptación del proyecto integrador previo a la última versión del sistema operativo. La actualización del proyecto no solo nos brinda acceso a nuevas funcionalidades avanzadas, sino que también nos permite aprovechar las mejoras en términos de seguridad implementadas en dicha versión. Además, las versiones anteriores eventualmente dejan de tener soporte por parte de la comunidad, dificultando la colaboración y el intercambio de conocimientos.\par

Planteadas las actualizaciones correspondientes, se abordan dos aspectos cruciales en la optimización de la gestión de recursos en sistemas operativos. En primer lugar, se examina el ``Encendido y Apagado de Procesadores'' como un primer paso en la implementación de soluciones de ahorro de energía a nivel de software, antes de considerar enfoques de hardware. La finalidad es permitir la gestión activa de la energía al apagar procesadores no utilizados, lo que conlleva a una reducción significativa en el consumo energético, contribuyendo a una mayor eficiencia operativa del sistema. En segundo lugar, se introduce el concepto de ``Monopolización de Núcleos'' como una estrategia para priorizar procesos críticos por sobre otros, minimizando así interrupciones no deseadas. La capacidad de asignar núcleos específicos a tareas es fundamental para garantizar un rendimiento consistente y predecible en diversos entornos. \par