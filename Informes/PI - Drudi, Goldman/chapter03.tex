\section{Desarrollo}

\subsection{Introducción}
En este capítulo, detallaremos el desarrollo de las mejoras implementadas en el planificador a corto plazo basado en Redes de Petri para el sistema operativo FreeBSD. El trabajo realizado se fundamenta en la colaboración y continuidad de proyectos previos desarrollados por estudiantes de nuestra institución.\par

La comprensión y depuración del código existente, así como la adaptación de los archivos del código fuente de FreeBSD relacionados con el planificador 4BSD, fueron tareas esenciales en nuestro proceso de desarrollo. Este esfuerzo intenta reflejar buenas prácticas en ingeniería de software, siendo ésta, una prioridad durante las primeras etapas de desarrollo. Nuestro enfoque integró principios de trabajo en equipo y metodologías ágiles, permitiendo una iteración continua y mejorada sobre el código. A lo largo del desarrollo, se implementaron prácticas de gestión de versiones y estrategias de ramificación para garantizar la calidad y estabilidad del código en cada una de sus etapas. En las siguientes secciones, se detallarán las metodologías utilizadas y los módulos específicos desarrollados, subrayando la importancia de cada componente en el contexto del planificador 4BSD.\par


\subsection{Metodologías de trabajo}
Para abordar el desarrollo de este proyecto de manera efectiva, establecimos una serie de metodologías de trabajo que nos permitieron mantener una organización rigurosa y una documentación detallada en cada etapa del proceso. Nuestro objetivo era no solo avanzar de manera sistemática, sino también crear un recurso valioso para futuros trabajos en este ámbito.

La primera etapa consistió en una profunda revisión del informe del trabajo integrador previo, así como del código base de los archivos de interés en el planificador 4BSD. Este proceso se realizó de forma iterativa, asegurándonos de comprender y extender la documentación de ambos componentes antes de proceder. Ampliar la documentación en esta etapa de análisis inicial, fue crucial para establecer una base sólida sobre la cual construir el resto del proyecto.

Una vez completada la fase de comprensión y documentación, planificamos nuestras metodologías de trabajo, enfocándonos en estrategias que facilitaran la colaboración y la gestión eficiente del proyecto. Implementamos una estrategia de ramas estandarizada en el repositorio fork de FreeBSD, permitiendo un control preciso de las versiones y facilitando la integración continua. Además, creamos un repositorio externo dedicado a almacenar toda la documentación y los recursos necesarios para el desarrollo, garantizando un acceso fácil y organizado a la información relevante.

Estas metodologías no solo nos permitieron trabajar de manera ordenada y eficiente, sino que también sentaron las bases para una colaboración efectiva y una fácil transición del conocimiento. En las siguientes secciones, se detallarán las estrategias específicas utilizadas.

\subsubsection{Estrategia de ramificación}
Definir una estrategia de ramificación o \textit{branch strategy} en los proyectos es fundamental para mantener el control sobre el flujo de trabajo, la organización del proyecto y asegurar que las modificaciones se integren sin problemas en la rama principal del código. Una estrategia bien definida establece un conjunto de reglas claras y consistentes para la creación y fusión de ramas de código.

Además, esta estrategia facilita el mantenimiento de un historial completo y bien organizado de las modificaciones realizadas al código, lo cual es esencial para el seguimiento y la continuidad del proyecto. En nuestro caso, definimos el prefijo de branches utilizando los apellidos de los integrantes del grupo de trabajo, como por ejemplo, \texttt{\textbf{DrudiGoldmanPI/}}. Este enfoque nos ayuda a identificar y localizar fácilmente todos los cambios realizados en este proyecto integrador.

Para definir los nombres de las ramas, establecimos tres casos principales por los cuales se crearían nuevas ramas:

\begin{itemize}
    \item \textbf{Testing}: Ramas creadas para realizar pruebas y verificar que ciertas partes del código fuente funcionen según lo esperado. Este tipo de ramas se utilizan principalmente para la depuración mediante el uso de logs.
    \item \textbf{Update}: Ramas dedicadas a la actualización del código base con las versiones más recientes de FreeBSD. Esto es especialmente relevante en la fase inicial del proyecto, cuando se deben integrar los cambios previos realizados en el proyecto integrador anterior.
    \item \textbf{Feature}: Ramas utilizadas para el desarrollo de nuevas funcionalidades o la adición de código no existente previamente. Estas ramas permiten la implementación y prueba de nuevas características sin afectar la estabilidad de la rama principal.
\end{itemize}

Con estos tres casos principales (o causas), se definió la estrategia de nombrado de ramas de la siguiente forma:

\begin{center}
    \texttt{\textbf{DrudiGoldmanPI/<CAUSA>\_<DESCRIPCIÓN>-<VERSION\_FREEBSD>}}
\end{center}

Donde \texttt{\textbf{<DESCRIPCIÓN>}} es una breve explicación del contenido de los cambios y \texttt{\textbf{<VERSION\_FREEBSD>}}, es la versión de FreeBSD que se utilizó para realizar los cambios.

Por ejemplo, \textit{DrudiGoldmanPI/feature\_cpuMonopolized-13.1.0} es una ramas para la implementación del módulo de monopolización que se detallará posteriormente en esta seccion de desarrollo; y \textit{DrudiGoldmanPI/update\_petriNetScheduler-12.3.0} es una rama para la actualización del planificador 4BSD mediante Redes de Petri a la versión 12.3.0 del sistema operativo.

\subsubsection{Conventional Commits}
En nuestro proyecto, utilizamos la convención de \textit{Conventional Commits} para mantener un historial de cambios claro y consistente. Esta práctica facilita la lectura y comprensión de los cambios realizados, tanto para el equipo de desarrollo como para futuros colaboradores.

Los mensajes de \textit{commit} siguen un formato estandarizado que describe de manera concisa el tipo de cambio, la versión del sistema operativo afectada y una breve descripción de la modificación. A continuación, se presentan algunos ejemplos extraídos de nuestro proyecto:

\begin{itemize}
    \item \texttt{feat(13.1.0): Move thread lock before resource\_expulse\_thread()}
    \item \texttt{fix(13.1.0): Fix sched\_pickcpu for cases when resource\_choose\_cpu returns -1 (NOCPU)}
    \item \texttt{refactor(13.1.0): Change kernel config file identifier}
    \item \texttt{feat(12.3.0): Short term scheduler modeling with PN - Add PI code to release/12.3.0}
\end{itemize}

Los tipos de cambios más comunes que utilizamos incluyen:
\begin{itemize}
    \item \textbf{feat}: Para la adición de nuevas funcionalidades.
    \item \textbf{fix}: Para la corrección de errores.
    \item \textbf{refactor}: Para modificaciones que mejoran el código sin alterar su funcionalidad.
\end{itemize}

Para la integración de cambios, hemos adoptado la estrategia de \textit{squash merge}. Esta técnica permite consolidar todos los cambios de una rama en un único \textit{commit} bien detallado, lo cual simplifica el historial del proyecto y facilita el seguimiento de avances importantes. Cada \textit{squash merge} incluye un resumen comprensivo de todos los \textit{sub-commits} que forman parte del cambio, garantizando así una documentación clara y precisa del progreso del proyecto.

\subsubsection{Repositorio de Documentación del Proyecto} \label{sect:wiki}

El repositorio de la documentación del proyecto está organizado de manera estructurada y detallada, facilitando el acceso y la gestión de la información relevante para el desarrollo del mismo. A continuación, se presenta una descripción breve de su estructura principal:

\begin{itemize}
    \item \textbf{Archivos del Proyecto}
    \begin{itemize}
        \item \textbf{Archivos de Migraciones:} Contiene subdirectorios correspondientes a diferentes versiones de migración del sistema, incluyendo archivos de configuración y código fuente relevante.
        \item \textbf{Logs:} Registro de eventos y mensajes generados durante pruebas y actualizaciones del sistema.
        \item \textbf{Módulos:} Módulos de kernel adicionales para poner en funcionamiento los nuevos desarrollos.
        \item \textbf{Otros:} Programas auxiliares para seguir el comportamiento de las transiciones de la Red de Petri.
        \item \textbf{Programas de Test:} Programas utilizados para pruebas de rendimiento y funcionalidad.
    \end{itemize}
    \item \textbf{Documentación del Proyecto}
    \begin{itemize}
        \item \textbf{Uso del Repositorio:} Guías sobre cómo manejar las ramas y versiones del repositorio.
        \item \textbf{Iteraciones:} Detalles de las distintas iteraciones del proyecto.
    \end{itemize}
\end{itemize}

Esta organización meticulosa asegura que todos los aspectos del proyecto, desde el código fuente hasta la documentación y los informes, estén bien gestionados y accesibles para los miembros del equipo.

Para más detalles, puede acceder al repositorio de documentación del proyecto en: \url{https://github.com/drudilea/free-bsd-scheduler}.
