\appendix

\section{Apéndice 1: Archivos de diferencias}\label{appendix:apA}

\subsection{\textit{sched\_switch} (v12 a v13): Firma de la función y variables}\label{appendix:apA1}

\begin{lstlisting}[language=diff]
- void sched_switch(struct thread *td, struct thread *newtd, int flags)
+ void sched_switch(struct thread *td, int flags)
{
+   struct thread *newtd;
    struct mtx *tmtx;
    struct td_sched *ts;
    struct proc *p;
    int preempted;

-   tmtx = NULL;
+   tmtx = &sched_lock;
    ts = td_get_sched(td);
    p = td->td_proc;

    THREAD_LOCK_ASSERT(td, MA_OWNED);
...
}
\end{lstlisting}



\subsection{\textit{sched\_switch} (v12 a v13): Cambio de posición del bloque que bloquea el hilo previo al \textit{resource\_expulse\_thread}}\label{appendix:apA2}

\begin{lstlisting}[language=diff]
+   if (td->td_lock != &sched_lock) {
+   	mtx_lock_spin(&sched_lock);
+   	tmtx = thread_lock_block(td);
+   	mtx_unlock_spin(tmtx);
+   }

   if ((td->td_flags & TDF_NOLOAD) == 0)
   	sched_load_rem();

    td->td_lastcpu = td->td_oncpu;
    preempted = (td->td_flags & TDF_SLICEEND) == 0 && (flags & SW_PREEMPT) != 0;
    td->td_flags &= ~(TDF_NEEDRESCHED | TDF_SLICEEND);
    td->td_owepreempt = 0;
    td->td_oncpu = NOCPU;

    resource_expulse_thread(td, flags);

-   if (td->td_lock != &sched_lock) {
-   	mtx_lock_spin(&sched_lock);
-   	tmtx = thread_lock_block(td);
-   	mtx_unlock_spin(tmtx);
-   }
...
\end{lstlisting}


\subsection{\textit{sched\_switch} (v12 a v13): Selección y ejecución de un nuevo thread}\label{appendix:apA3}

\begin{lstlisting}[language=diff]
+   newtd = choosethread();
-   if (newtd) {
-       /*
-        * The thread we are about to run needs to be counted
-        * as if it had been added to the run queue and selected.
-        * It came from:
-        * * A preemption
-        * * An upcall
-        * * A followon
-        */
-       KASSERT((newtd->td_inhibitors == 0),
-           ("trying to run inhibited thread"));
-       newtd->td_flags |= TDF_DIDRUN;
-           TD_SET_RUNNING(newtd);
-       if ((newtd->td_flags & TDF_NOLOAD) == 0)
-           sched_load_add();
-       if (ts->ts_runq != &runq){
-           resource_fire_net(newtd, TRAN_UNQUEUE + (PCPU_GET(cpuid)*CPU_BASE_TRANSITIONS));
-       }
-       else{
-           resource_fire_net(newtd, TRAN_FROM_GLOBAL_CPU + (PCPU_GET(cpuid)*CPU_BASE_TRANSITIONS));
-       }
-   } else {
-       newtd = choosethread();
-       MPASS(newtd->td_lock == &sched_lock);
-   }

    resource_execute_thread(newtd, PCPU_GET(cpuid));

\end{lstlisting}

% TODO: ANEXO
\section{Apéndice 2: Detalles de la implementación del módulo de monopolizado}\label{appendix:apB}

El planificador 4BSD maneja los cambios de contexto cuando un hilo termina su ejecución o queda bloqueado, generando una interrupción para seleccionar un nuevo hilo. La función principal para gestionar estos cambios es \textit{mi\_switch()}, que invoca \textit{sched\_switch()}. Esta función se encarga de preparar el nuevo hilo para su ejecución y usa \textit{sched\_add()} para añadirlo a la cola de un procesador, considerando la afinidad y las políticas del planificador. Luego, \textit{sched\_choose()} selecciona el hilo a ejecutar y realiza el cambio de contexto si es necesario.

% Para integrar el módulo de monopolización, modificamos \textit{sched\_add()} para permitir que ciertos hilos se fijen a CPUs específicas. Esto asegura que un CPU ejecute siempre el mismo hilo hasta que se decida lo contrario. Implementamos un arreglo llamado \textit{pinned\_threads\_per\_cpu}, que almacena el ID del hilo asignado a cada CPU. Si un CPU está libre, el valor correspondiente en el arreglo es -1. Este ajuste facilita la asignación precisa de hilos a CPUs y asegura que cada procesador ejecute los hilos de manera eficiente según las políticas establecidas.

% En el siguiente ejemplo, se detalla un caso en el que el hilo con ID 100101 tomó control sobre el CPU1 y donde el resto de los procesadores se encuentran funcionando normalmente.

% int \textit{pinned\_threads\_per\_cpu}[CPU\_NUMBER] = \{ -1, 100101, -1, -1 \};

% Al momento de encolar el hilo, se busca cuál de los procesadores disponibles sería la mejor opción para continuar la ejecución del mismo. Ésta decisión se toma dentro del método resource\_choose\_cpu desarrollado en el trabajo integrador previo y extendido actualmente, para hacer uso de este nuevo arreglo de hilos asociados a procesadores. Recibe como parámetro el hilo que se encuentra a encolar, y cuenta con tres condicionales que determinarán dicho procesador:

% \begin{itemize}
%     \item Como primera condición, si el hilo se encuentra dentro del arreglo de \textit{pinned\_threads\_per\_cpu} ya estamos en condiciones de elegir dicho procesador como el indicado para el encolado.
%     \item Si el hilo no se encuentra dentro del arreglo, se intenta asignar a la cola del último procesador en el que se ejecutó. Esto es posible solo si la transición TRAN\_ADDTOQUEUE de dicho procesador se encuentra sensibilizada y el procesador no está monopolizado por otro hilo.
%     \item Por último, si no se cumplen ninguna de las dos condiciones previas, se recorren los diferentes núcleos del procesador y se retorna el primero que cumpla las condiciones necesarias para el encolado.
% \end{itemize}

% Como parte del desarrollo del módulo, también se implementaron algunos métodos complementarios encargados de manejar el estado del arreglo  \textit{pinned\_threads\_per\_cpu}.

% \begin{itemize}
%     \item toggle\_pin\_thread\_to\_cpu: Método encargado de conmutar el estado de monopolización de un procesador. Recibe el ID de un hilo y de un procesador como parámetros y realiza diferentes operaciones de acuerdo al estado del arreglo:
%     \begin{itemize}
%         \item Si el procesador se encuentra libre, se agrega el ID del hilo a la posición correspondiente en el arreglo.
%         \item Si el procesador ya estaba monopolizado por otro hilo, se sobreescribe con el nuevo ID.
%         \item En caso de que el hilo ya se encuentre monopolizando al procesador, lo libera escribiendo el valor -1 en la posición correspondiente.
%     \end{itemize}
%     \item cpu\_available\_for\_thread: Método utilizado por resource\_choose\_cpu para saber si un hilo puede utilizar un procesador, o si este se encuentra monopolizado por otro. Recibe el ID de un hilo y de un procesador como parámetros y retorna 1 en caso de que el procesador se encuentre habilitado para encolar dicho hilo; retorna 0 en caso de que dicho procesador se encuentre tomado por otro hilo.
%     \item get\_monopolized\_cpu\_by\_thread\_id: Método utilizado por resource\_choose\_cpu para obtener el ID del procesador al que se encuentra asociado el hilo enviado por parámetro. Retorna -1 en caso de que no esté anclado a ningún CPU.
% \end{itemize}








\end{document}