\subsection{Módulo de encendido/apagado de los procesadores}

Tras completar las actualizaciones del sistema operativo, el siguiente paso fue desarrollar una nueva funcionalidad enfocada en el control de los núcleos del procesador. Esta funcionalidad está diseñada para proporcionar una herramienta flexible que permita la gestión dinámica del encendido y apagado de los núcleos del procesador, con la expectativa de que futuros proyectos utilicen este módulo para optimizar el ahorro energético.

Este módulo es especialmente útil en escenarios donde la carga de trabajo varía significativamente, permitiendo controlar los núcleos durante periodos de baja demanda y ajustarlos cuando la carga aumenta. Aunque aún no se ha implementado una estrategia de ahorro energético, esta herramienta abre la puerta a desarrollos futuros que podrían aprovechar esta capacidad para reducir el consumo de energía y mejorar la eficiencia del sistema.

\subsubsection{Objetivos}

El objetivo general de esta sección, fue el desarrollo de una nueva funcionalidad dentro del \textit{scheduler} del sistema operativo FreeBSD que permita habilitar o deshabilitar los diferentes núcleos del procesador de manera dinámica.

Este objetivo principal se traduce en múltiples aplicaciones que pueden proporcionar ventajas significativas, tales como la optimización del uso de recursos de hardware bajo cargas de trabajo reducidas, la restricción de núcleos disponibles para aplicaciones específicas, y la capacidad de ajustar el estado de los procesadores en tiempo de ejecución.

\subsubsection{Primera Iteración: Implementación del Módulo de Encendido/Apagado mediante cambios en la Red de Petri}

Para iniciar la implementación del módulo de encendido/apagado de núcleos del procesador, se debatieron dos enfoques relacionados con la Red de Petri del sistema operativo. El primer enfoque proponía utilizar las plazas y transiciones ya existentes, estableciendo diferentes políticas a través de cambios en el código del \textit{scheduler}. El segundo enfoque, que fue el adoptado finalmente, implicaba la incorporación de nuevas plazas y transiciones en la Red de Petri para representar explícitamente el estado de los núcleos del procesador. Optamos por esta segunda opción porque nos permitió integrar el módulo de manera más fluida al sistema ya existente, sin hacer cambios drásticos en el núcleo del sistema operativo, y ampliando al mismo tiempo las capacidades de la Red de Petri. La principal ventaja de esta implementación, y motivador principal, fue que desde cualquier parte del sistema, simplemente disparando una transición, fuera posible activar o desactivar cualquier núcleo del procesador según sea necesario.

Con esta decisión en mente, el primer paso fue crear un indicador que permitiera conocer si cada procesador está encendido o apagado. Para esto, se añadió una nueva plaza en la Red de Petri, llamada \texttt{PLACE\_SUSPENDED}. La presencia o ausencia de un \textit{token} en esta plaza indica si un procesador está disponible para encolar y ejecutar hilos.

La plaza \texttt{PLACE\_SUSPENDED} está acompañada de dos transiciones que manejan el \textit{token} correspondiente: \texttt{TRAN\_SUSPEND\_PROC} para deshabilitar el núcleo, y \texttt{TRAN\_WAKEUP\_PROC} para habilitarlo nuevamente. Se añadió un arco inhibidor a la transición que libera el recurso para asegurar que la plaza funcione como un recurso único. Una vez que un procesador está deshabilitado, solo la transición que lo habilita de nuevo queda sensibilizada.

En resumen, la plaza \texttt{PLACE\_SUSPENDED} controla las transiciones de encolado, tanto de la cola específica de la CPU, como de la cola global. Cuando la CPU está suspendido, impide que nuevos hilos se añadan a su cola y se ejecuten. Sin embargo, es importante destacar que este desarrollo no elimina los hilos que ya estaban en la cola de la CPU antes de que se deshabilitara. Es decir, los hilos ya encolados seguirán siendo ejecutados, pero no podrán volver a la cola de la CPU inhabilitado. Esta lógica está reflejada en la Red de Petri, donde la nueva plaza de suspensión solo afecta a las transiciones de encolado y desencolado, sin interferir en la ejecución de hilos ni en la transición específica de desencolado del procesador (\texttt{TRAN\_UNQUEUE}).

\begin{figure}[H]
    \centering
    \vspace*{0.1in}
    \includegraphics[width=0.8\textwidth]{./images/cpuOnOff-1st-iteration.png}
    \caption{Primera iteración del módulo de encendido/apagado de núcleos del procesador.}
    \label{fig:cpu-on-off-1st-iteration}
\end{figure}

Como primer paso para implementar esta funcionalidad a nivel de código, nos centramos en ajustar las matrices fundamentales que soportan el funcionamiento de la Red de Petri: la matriz base y la matriz de incidencia. Estas matrices son cruciales para la representación y el manejo de las transiciones y plazas en la Red de Petri.\par

La matriz base, que define la estructura inicial de la Red de Petri, necesitaba ser actualizada para reflejar las nuevas plazas y transiciones introducidas. Específicamente, agregamos entradas para la nueva plaza \texttt{PLACE\_SUSPENDED} y las transiciones asociadas (\texttt{TRAN\_SUSPEND\_PROC} y \texttt{TRAN\_WAKEUP\_PROC}). Esta modificación asegura que la estructura de la red pueda manejar los cambios en el estado de los núcleos del procesador, permitiendo que el sistema registre y gestione correctamente los \textit{tokens} en la plaza de suspensión.\par

\renewcommand{\arraystretch}{1.5}
\setlength{\tabcolsep}{10pt}

\begin{table}[H]
    \centering
    \begin{tabular}{|c|c|c|c|c|c|c|c|c|}
        \hline
        1 & 0  & -1 & 0  & 0  & 0  & 0  & -1 & 0  \\
        \hline
        1 & -1 & 0  & 0  & 0  & 0  & 0  & -1 & -1 \\
        \hline
        0 & -1 & 0  & 0  & 1  & 1  & -1 & 0  & 0  \\
        \hline
        0 & 1  & -1 & -1 & 0  & 0  & 1  & 0  & 0  \\
        \hline
        0 & 0  & 1  & 1  & -1 & -1 & 0  & 0  & 0  \\
        \hline
    \end{tabular}
    \caption{Matriz base inicial.}
    \label{tabla:matriz_base_pre}
\end{table}

\begin{table}[H]
    \centering
    \begin{tabular}{|c|c|c|c|c|c|c|c|c|c|c|}
        \hline
        1                      & 0                      & -1                     & 0                      & 0                      & 0                      & 0                      & -1                     & 0                      & \cellcolor{lightgray}0 & \cellcolor{lightgray}0  \\
        \hline
        1                      & -1                     & 0                      & 0                      & 0                      & 0                      & 0                      & -1                     & -1                     & \cellcolor{lightgray}0 & \cellcolor{lightgray}0  \\
        \hline
        0                      & -1                     & 0                      & 0                      & 1                      & 1                      & -1                     & 0                      & 0                      & \cellcolor{lightgray}0 & \cellcolor{lightgray}0  \\
        \hline
        0                      & 1                      & -1                     & -1                     & 0                      & 0                      & 1                      & 0                      & 0                      & \cellcolor{lightgray}0 & \cellcolor{lightgray}0  \\
        \hline
        0                      & 0                      & 1                      & 1                      & -1                     & -1                     & 0                      & 0                      & 0                      & \cellcolor{lightgray}0 & \cellcolor{lightgray}0  \\
        \hline
        \cellcolor{lightgray}0 & \cellcolor{lightgray}0 & \cellcolor{lightgray}0 & \cellcolor{lightgray}0 & \cellcolor{lightgray}0 & \cellcolor{lightgray}0 & \cellcolor{lightgray}0 & \cellcolor{lightgray}0 & \cellcolor{lightgray}0 & \cellcolor{lightgray}1 & \cellcolor{lightgray}-1 \\
        \hline
    \end{tabular}
    \caption{Matriz base: Primera iteración del módulo de encendido/apagado.}
    \label{tabla:matriz_base_post}
\end{table}

Por otro lado, la matriz de incidencia describe cómo las transiciones afectan a las plazas y viceversa. Actualizar esta matriz implicó modificar las relaciones entre las nuevas transiciones y las plazas afectadas. Concretamente, tuvimos que ajustar las entradas para reflejar el impacto de las transiciones de habilitación y deshabilitación de los núcleos en las plazas correspondientes. Esto incluye la configuración de los arcos inhibidores que garantizan que los cambios en el estado del procesador se gestionen correctamente.\par

\renewcommand{\arraystretch}{1.5}
\setlength{\tabcolsep}{10pt}

\begin{table}[H]
    \centering
    \begin{tabular}{|c|c|c|c|c|c|c|c|c|}
        \hline
        1 & 0 & 0 & 1 & 0 & 0 & 0 & 0 & 1 \\
        \hline
        0 & 0 & 0 & 0 & 0 & 0 & 0 & 0 & 0 \\
        \hline
        0 & 0 & 0 & 0 & 0 & 0 & 0 & 0 & 0 \\
        \hline
        0 & 0 & 0 & 0 & 0 & 0 & 0 & 0 & 0 \\
        \hline
        0 & 0 & 0 & 0 & 0 & 0 & 0 & 0 & 0 \\
        \hline
    \end{tabular}
    \caption{Matriz de incidencia inicial.}
    \label{tabla:matriz_incidencia_pre}
\end{table}

\begin{table}[H]
    \centering
    \begin{tabular}{|c|c|c|c|c|c|c|c|c|c|c|}
        \hline
        1                      & 0                      & 0                      & 1                      & 0                      & 0                      & 0                      & 0                      & 1                      & \cellcolor{lightgray}0 & \cellcolor{lightgray}0 \\
        \hline
        0                      & 0                      & 0                      & 0                      & 0                      & 0                      & 0                      & 0                      & 0                      & \cellcolor{lightgray}0 & \cellcolor{lightgray}0 \\
        \hline
        0                      & 0                      & 0                      & 0                      & 0                      & 0                      & 0                      & 0                      & 0                      & \cellcolor{lightgray}0 & \cellcolor{lightgray}0 \\
        \hline
        0                      & 0                      & 0                      & 0                      & 0                      & 0                      & 0                      & 0                      & 0                      & \cellcolor{lightgray}0 & \cellcolor{lightgray}0 \\
        \hline
        0                      & 0                      & 0                      & 0                      & 0                      & 0                      & 0                      & 0                      & 0                      & \cellcolor{lightgray}0 & \cellcolor{lightgray}0 \\
        \hline
        \cellcolor{lightgray}1 & \cellcolor{lightgray}0 & \cellcolor{lightgray}0 & \cellcolor{lightgray}0 & \cellcolor{lightgray}0 & \cellcolor{lightgray}0 & \cellcolor{lightgray}1 & \cellcolor{lightgray}0 & \cellcolor{lightgray}0 & \cellcolor{lightgray}1 & \cellcolor{lightgray}0 \\
        \hline
    \end{tabular}
    \caption{Matriz de incidencia: Primera iteración del módulo de encendido/apagado.}
    \label{tabla:matriz_incidencia_post}
\end{table}

Como parte de los ajustes realizados en las matrices, se incorporó una nueva función, \textit{toggle\_active\_cpu}, al archivo de métodos auxiliares de la Red de Petri. Esta función puede ser invocada desde un módulo del \textit{kernel} y está diseñada para alternar el estado de un CPU especificado como parámetro. La función verifica si la transición asociada está sensibilizada y, si es así, procede a dispararla. En resumen, esta adición permite seleccionar y alternar el estado de un procesador, que puede ser suspendido o activado según sea necesario, y se gestiona de manera externa al flujo principal del kernel, permitiendo así un control en tiempo real del estado de habilitación de la CPU.\par

Tras la implementación de esta funcionalidad, nos encontramos con un problema inesperado: la funcionalidad no se comportaba como se había previsto. Específicamente, recibimos alertas indicando que algunas transiciones no estaban sensibilizadas durante los intentos de disparo. El proceso de depuración reveló que el problema estaba relacionado con el comportamiento del \textit{scheduler} cuando no había hilos disponibles para ejecutar, ni en la cola local del procesador ni en la cola global.

En la siguiente sección, se describe la solución implementada para superar esta dificultad.

\subsubsection{Segunda Iteración: Soporte del \textit{idlethread} en la Red de Petri}
\label{ch:idlethread}

Durante la depuración realizada en la iteración anterior, se identificó un error en situaciones donde una CPU no tenía hilos listos para ejecutar. Para abordar este problema en la segunda iteración del módulo de encendido/apagado, es importante entender el papel del \textit{idlethread}. En estos casos, el planificador encola y ejecuta el \textit{idlethread}, una funcionalidad que ya estaba presente antes del inicio del proyecto integrador y que es característica del planificador 4.4BSD utilizado en FreeBSD.\par

El \textit{idlethread} es un hilo \textit{dummy}, es decir, un hilo que no genera carga significativa para el procesador. Puede ser encolado, suspendido y ejecutado de manera similar a otros hilos. Cada núcleo de CPU tiene asociado su propio \textit{idlethread}, el cual ejecuta un bucle simple que generalmente consiste en poner la CPU en un estado de bajo consumo.

Cuando no hay hilos o procesos listos para ejecutarse en una CPU, el planificador selecciona el \textit{idlethread} como el siguiente hilo. Dado que el \textit{idlethread} tiene la prioridad más baja, solo se programa cuando no hay tareas de mayor prioridad disponibles. En cualquier momento en que aparezca un hilo de mayor prioridad, el \textit{idlethread} será reemplazado por este nuevo hilo.

En el trabajo integrador previo, el momento de la toma de decisión del encolado del \textit{idlethread} no estaba modelado en su totalidad, sino que se simulaban los disparos de transiciones utilizando la cola global, para mantener coherentes los estados en ambas redes (de recursos y de hilos). Es decir, en momentos de inactividad de la CPU, se disparaban las transiciones de \texttt{TRAN\_QUEUE\_GLOBAL} y \texttt{TRAN\_FROM\_GLOBAL\_CPU} para encolar, desencolar y ejecutar el \textit{idlethread}.

El problema con este enfoque en nuestro proyecto es que la decisión de inhabilitar el procesador mediante el modulo desarrollado, bloquea tanto el encolado como el desencolado, ya sea de la cola particular del procesador como de la cola global, lo que hace que el mecanismo utilizado en el proyecto integrador previo quede inutilizable, ya que se intentarán disparar transiciones no sensibilizadas. Por tanto, fue necesario ajustar el modelo de la red de Petri para incluir ahora el comportamiento específico del planificador en los momentos en que se tiene que hacer uso del \textit{idlethread}.

La solución, detallada en la Figura \ref{fig:cpu-on-off-2nd-iteration}, consistió en agregar una nueva transición, \texttt{TRAN\_EXEC\_IDLE}, que permite la ejecución del \textit{idlethread} cuando no hay hilos en la cola del procesador, contemplando así el caso en que la CPU está suspendida. Tras implementar estos cambios, el modelo comenzó a representar con precisión el proceso interno del planificador.\par

\begin{figure}[H]
    \centering
    \vspace*{0.1in}
    \includegraphics[width=0.8\textwidth]{./images/cpuOnOff-2nd-iteration.png}
    \caption{Segunda iteración del módulo de encendido/apagado de núcleos del procesador.}
    \label{fig:cpu-on-off-2nd-iteration}
\end{figure}

Una vez más, la implementación de estos cambios comenzó con la actualización de las matrices base y de incidencia. Después de realizar estos ajustes, se pudo proceder a incorporar las modificaciones correspondientes en el código del planificador. Dichos cambios incluyeron la implementación de la lógica detallada previamente, donde en casos de inactividad o inhabilitación manual de algún procesador, el flujo de ejecución de la red y, por tanto, de los procesos e hilos dentro de un núcleo, se comportará de manera esperada.\par

\begin{table}[H]
    \centering
    \begin{tabular}{|c|c|c|c|c|c|c|c|c|c|c|c|}
        \hline
        1 & 0  & -1 & 0  & \cellcolor{lightgray}0  & 0  & 0  & 0  & -1 & 0  & 0 & 0  \\
        \hline
        1 & -1 & 0  & 0  & \cellcolor{lightgray}0  & 0  & 0  & 0  & -1 & -1 & 0 & 0  \\
        \hline
        0 & -1 & 0  & 0  & \cellcolor{lightgray}-1 & 1  & 1  & -1 & 0  & 0  & 0 & 0  \\
        \hline
        0 & 1  & -1 & -1 & \cellcolor{lightgray}1  & 0  & 0  & 1  & 0  & 0  & 0 & 0  \\
        \hline
        0 & 0  & 1  & 1  & \cellcolor{lightgray}0  & -1 & -1 & 0  & 0  & 0  & 0 & 0  \\
        \hline
        0 & 0  & 0  & 0  & \cellcolor{lightgray}0  & 0  & 0  & 0  & 0  & 0  & 1 & -1 \\
        \hline
    \end{tabular}
    \caption{Matriz base: Segunda iteración del módulo de encendido/apagado.}
    \label{tabla:matriz_base_post_2}
\end{table}

\begin{table}[H]
    \centering
    \begin{tabular}{|c|c|c|c|c|c|c|c|c|c|c|c|}
        \hline
        1 & 0 & 0 & 1 & \cellcolor{lightgray}0 & 0 & 0 & 0 & 0 & 1 & 0 & 0 \\
        \hline
        0 & 0 & 0 & 0 & \cellcolor{lightgray}1 & 0 & 0 & 0 & 0 & 0 & 0 & 0 \\
        \hline
        0 & 0 & 0 & 0 & \cellcolor{lightgray}0 & 0 & 0 & 0 & 0 & 0 & 0 & 0 \\
        \hline
        0 & 0 & 0 & 0 & \cellcolor{lightgray}0 & 0 & 0 & 0 & 0 & 0 & 0 & 0 \\
        \hline
        0 & 0 & 0 & 0 & \cellcolor{lightgray}0 & 0 & 0 & 0 & 0 & 0 & 0 & 0 \\
        \hline
        1 & 0 & 0 & 0 & \cellcolor{lightgray}0 & 0 & 0 & 1 & 0 & 0 & 1 & 0 \\
        \hline
    \end{tabular}
    \caption{Matriz de incidencia: Segunda iteración del módulo de encendido/apagado.}
    \label{tabla:matriz_incidencia_post_2}
\end{table}

\subsubsection{Resultados}

Con la finalización del desarrollo de esta nueva funcionalidad, se procedió a realizar las pruebas correspondientes.\par

En resumen, la funcionalidad propuesta para el planificador, que permite habilitar o inhabilitar núcleos del procesador, ha sido implementada con éxito, proporcionando un control que antes no estaba disponible.\par

Además, se logró la versatilidad esperada para este módulo, ya que es posible activar esta mejora en tiempo real sin necesidad de reiniciar el sistema operativo.\par

Los detalles completos de los resultados se discutirán en profundidad en el capítulo \ref{ch:results} de \textit{Análisis de Resultados}.\par

Es importante señalar que no es posible suspender el procesador cero. Esto se debe a que el sistema operativo utiliza este núcleo de forma predeterminada para tareas esenciales como la gestión de interrupciones, la gestión de memoria, entre otras; y al desactivarlo, el sistema experimenta un comportamiento inesperado. Para evitar esta situación, se realizaron ajustes en el código que impiden la selección de dicho núcleo para su suspensión.\par

\subsubsection{Próximos pasos}

La implementación de este módulo independiente no tiene un impacto inmediato en el sistema, ya que se activa únicamente cuando el usuario envía una señal específica, en el momento que elija.\par

De cara al futuro, creemos que el siguiente paso debería enfocarse en integrar esta funcionalidad con otros elementos clave del sistema operativo, como hilos, procesos o programas que los gestionen. Esta integración permitiría ajustar el estado de los procesadores en tiempo real, respondiendo a las necesidades reales del sistema en su conjunto, y no solo a través de un módulo del kernel.\par

Otra mejora, relacionada con la estructura de la red de Petri de recursos desarrollada en la sección \ref{ch:idlethread}, sería asignar una plaza específica en la red para el \textit{idlethread}. Esto facilitaría la detección de los momentos en los que el \textit{idlethread} está en ejecución, proporcionando un mayor control sobre posibles acciones en estas situaciones.