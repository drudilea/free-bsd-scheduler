\subsection{Módulo de monopolización de hilos por parte de los procesadores}

Tras finalizar el módulo de encendido y apagado de procesadores, nos enfocamos en el desarrollo del módulo de monopolización de hilos por parte de los procesadores. Este módulo introduce la capacidad en el planificador de anclar hilos a una CPU específica, lo que implica que dicho núcleo solo puede encolar y ejecutar los hilos que están asociados a él.\par

\subsubsection{Objetivos}

El objetivo de este módulo es proporcionar al planificador 4BSD la capacidad de anclar hilos a procesadores específicos en cualquier momento durante la ejecución del sistema operativo. Esto permitirá definir políticas de asignación de recursos más precisas y adaptables a las necesidades del sistema.\par

Es muy importante que este módulo, al igual que el desarrollado previamente, pueda ser cargado y activado en tiempo real, sin necesidad de reiniciar el sistema operativo. Esta flexibilidad facilitará su integración con funcionalidades del sistema operativo, como la adaptación dinámica a cambios en la carga de trabajo o la asignación de hilos críticos a núcleos específicos para mejorar el rendimiento, entre otras.\par

% TODO: Revisar el nombre
\subsubsection{Primera Iteración: Desarrollo del Módulo de Monopolización a través de Políticas en la Red}

Para implementar el módulo de monopolización, se comenzó con una fase de investigación para explorar las posibles estrategias de desarrollo. Inicialmente, se consideró la opción de diseñarlo como un módulo independiente, similar al módulo de encendido/apagado. Sin embargo, se identificó que en última instancia, la decisión del planificador sobre a qué CPU encolar un hilo, estaba estrechamente vinculada a los identificadores de los hilos que se deseaban ejecutar.

Este enfoque nos llevó a entender el problema en términos de políticas para la toma de decisiones a la hora de encolar un hilo. Para implementar estas nuevas politicas, agregamos un mecanismo que permite fijar un hilo a un procesador específico, asegurandonos que dicho procesador no pueda encolar ni ejecutar otros hilos mientras esté monopolizado.

El proceso utiliza un nuevo registro (\textit{pinned\_threads\_per\_cpu}) que rastrea el estado de monopolización de cada uno de los procesadores, que se define como un arreglo de tantos elementos como procesadores en el sistema, y que cada elemento almacena el identificador del hilo que se encuentra monopolizando dicho procesador. Así, cuando el planificador debe decidir donde ejecutar el hilo, primero verifica si este se encuentra \textit{anclado} a un procesador específico. Si es así, el hilo se encola en ese procesador para luego ser ejecutado. En caso contrario, el sistema sigue el proceso estándar de asignación a cualquier procesador disponible.

Junto con el desarrollo de este sistema, también desarrollamos funciones complementarias que permiten alternar el estado de monopolización de los procesadores, liberar un procesador cuando se decida, y verificar si un procesador está ocupado o disponible para un nuevo hilo.

Los detalles técnicos y las implementaciones específicas se encuentran documentados en el anexo \ref{appendix:apB}.

% TODO: Agregar alguna imagen bien simple que explique la nueva forma de eleccion de cpu a encolar los hilos monopolizados.

\subsubsection{Resultados}

Los resultados del módulo fueron exitosos, logrando que los hilos pudieran monopolizar procesadores de manera efectiva.

Para comprobar este comportamiento, realizamos pruebas utilizando una herramienta de monitoreo y un programa de estrés. En estas pruebas, se generaron varios subprocesos que se ejecutaron en los distintos núcleos del sistema durante un tiempo prolongado. Posteriormente, seleccionamos uno de estos hilos y lo vinculamos a un procesador específico. Esto nos permitió observar cómo el hilo permanecía continuamente en el procesador al que se había anclado, sin que este ejecutara otros hilos no vinculados.

Como parte del resultado, también aclaramos que no se encuentra contemplado el caso del CPU0 en este módulo, ya que al igual que en el modulo de encendido/apagado, nos trajo problemas a la hora de la monopolización, debido a que es el encargado de manejar algunas tareas de administración del sistema.

Los detalles de los resultados serán explicados con mayor profundidad en el capítulo de
“Análisis de resultados”.

\subsubsection{Próximos pasos}

Al igual que en el módulo previo, la implementación de este módulo no tiene mejoras o impacto inmediato en el sistema; esto es debido a que son módulos completamente independientes que son activados por señales o por alguna otra parte del kernel.

Puede ser especialmente útil en escenarios en los que se necesite garantizar la ejecución de tareas críticas en tiempo real o cuando ciertos hilos requieran una capacidad de procesamiento dedicada y preferente.

Además de la priorización manual de hilos, esta modificación puede tener otras funcionalidades relacionadas. Por ejemplo, podría permitir la asignación de hilos a procesadores específicos según criterios como la afinidad de memoria o la afinidad de caché, optimizando así el rendimiento del sistema. También podría ser utilizado en entornos de computación distribuida, donde se necesite asignar tareas a CPUs específicas para aprovechar recursos especializados.

% TODO: Agregar como proximo paso la posibilidad de anclar multiples hilos a un procesador.